\documentclass[11pt]{articulo}

\textwidth 16.5cm 
\textheight 25.5cm
\topmargin -2cm
\oddsidemargin 0cm   
\evensidemargin 0cm  

%%%\input option_keys
%%%\input Definir(99)

\baselineskip0.6cm plus 1pt minus 1pt

 
 
\begin{document}
\begin{verse}
{\Large \it  UNIVERSIDAD DE CANTABRIA}\\ 
\vspace*{0.5cm}
{\normalsize \rm $2^{\underline \rm o}$ CURSO DE GRADO EN F\'ISICA.  2019-2020 }\\
{\normalsize \rm  Laboratorio de F\'isica I}\\ 
{\normalsize \rm  {\bf CUESTIONARIO. Resonancia. } Profesor J\'onatan Piedra}\\
{\normalsize \rm  TIEMPO: 10 minutos}\\
\end{verse} 

\vspace*{0.25cm}

NOMBRE (en may\'usculas) .................................................................................................................\\

 
\hrule

{\large \begin{enumerate}
\item Solo una de las respuestas por pregunta es correcta. 
\item Marque con un aspa sobre el cuadrado izquierdo s\'olo una res\-puesta por pregunta. 
\item No se permiten ni tachaduras ni enmiendas.
\end{enumerate}}
 

\hrule
          

\begin{enumerate}

\item La evoluci\'on temporal $x(t)$ de un sistema en {\bf equilibrio estable} al ser perturbado, puede ser descrita mediante la ecuaci\'on
\begin{enumerate}
\item[(a)] \fbox{$\, \phantom{-}\, $} \hspace*{1cm}   $x(t) = A \exp\, \left(\phantom{-}\omega t + \phi\right)$
\item[(b)] \fbox{$\, \phantom{-}\, $} \hspace*{1cm}   $x(t) = A \ln\, \phantom{s}\left(-\omega t + \phi\right)$
\item[(c)] \fbox{$\, \phantom{-}\, $} \hspace*{1cm}   $x(t) = A\,  \cos\, \left(\phantom{-}\omega t + \phi\right)$
\end{enumerate}

\item La evoluci\'on  temporal $x(t)$  de un sistema en {\bf equilibrio inestable} al ser perturbado, puede ser descrita mediante la ecuaci\'on
\begin{enumerate}
\item[(a)] \fbox{$\, \phantom{-}\, $} \hspace*{1cm}   $x(t) = A \exp\, \left(-\omega t + \phi\right)$
\item[(b)] \fbox{$\, \phantom{-}\, $} \hspace*{1cm}   $x(t) = A \, \cos\, \left(\phantom{-}\omega t + \phi\right)$
\item[(c)] \fbox{$\, \phantom{-}\, $} \hspace*{1cm}   ninguna de las anteriores
\end{enumerate}


\item Un {\bf p\'endulo matem\'atico} de longitud $L=0,25$~m, tendr\'a un per\'iodo de oscilaci\'on $T_{\rm f}$ de:
\begin{enumerate}
\item[(a)] \fbox{$\, \phantom{-}\, $} \hspace*{1cm}   $T_{\rm f} \approx \, \phantom{0,}2$~s
\item[(b)] \fbox{$\, \phantom{-}\, $} \hspace*{1cm}   $T_{\rm f} \approx \, \phantom{0,}1$~s
\item[(c)] \fbox{$\, \phantom{-}\, $} \hspace*{1cm}   $T_{\rm f} \approx \, 0,5$~s
\end{enumerate}


\item  Sea un {\bf p\'endulo de gravedad variable},  con per\'iodo $T_0$   cuando se encuentra vertical 
(con $\theta_0 = 0^\circ$). Para un   \'angulo de inclinaci\'on $\theta$ se tiene   el per\'iodo $T(\theta)=2T_0$. Entonces debe ser 
\begin{enumerate}
\item[(a)] \fbox{$\, \phantom{-}\, $} \hspace*{1cm}   $\theta \approx 30^\circ$
\item[(b)] \fbox{$\, \phantom{-}\, $} \hspace*{1cm}   $\theta\approx 45^\circ$
\item[(c)] \fbox{$\, \phantom{-}\, $} \hspace*{1cm}   $\theta > 60^\circ$
\end{enumerate}


%\item  Un {\bf p�ndulo f�sico}, una barra homog�nea de longitud  $L$,  oscila, en vertical, alrededor de un eje horizontal que pasa por uno de sus extremos.  El per�odo de oscilaci�n de la barra   $T_{\rm B}$   viene dado por:
%\begin{enumerate}
%\item[(a)] \fbox{$\, \phantom{-}\, $} \hspace*{1cm}   $T_{\rm B}=2\pi \left({2 L \over 3g}\right)^{-1/2}$
%\item[(b)] \fbox{$\, \phantom{-}\, $} \hspace*{1cm}   $T_{\rm B}=2\pi \left({3 L \over 2g}\right)^{1/2}$
%\item[(c)] \fbox{$\, \phantom{-}\, $} \hspace*{1cm}   $T_{\rm B}=2\pi \left({2 L\over 3g}\right)^{1/2}$
%\end{enumerate}

\item  Un p\'endulo f\'isico, un {\bf  disco homog\'eneo} de radio  $R$,  oscila, en vertical, alrededor de un eje horizontal que pasa por uno de los puntos de su circunferencia.  El per\'iodo de oscilaci\'on de este disco   $T_{\rm D}$   viene dado por:
\begin{enumerate}
\item[(a)] \fbox{$\, \phantom{-}\, $} \hspace*{1cm}   $T_{\rm D}=2 \pi \sqrt{2R/3 \over g}$
\item[(b)] \fbox{$\, \phantom{-}\, $} \hspace*{1cm}   $T_{\rm D}=2 \pi \sqrt{R  \over g}$
\item[(c)] \fbox{$\, \phantom{-}\, $} \hspace*{1cm}   $T_{\rm D}=2 \pi \sqrt{3R/2 \over g}$
\end{enumerate}


%T_{\rm A} = 2 \pi \sqrt{3R/2 \over g}

\newpage


\item  Un tubo de vidrio, de secci\'on $A$,  {\bf lastrado con bolitas de plomo}, con masa total $m$,  flota,  en vertical, sobre agua, de densidad $\rho_{\rm A}$.
El per\'iodo de oscilaci\'on del tubo $T_{\rm T}$ viene dado por:
\begin{enumerate}
\item[(a)] \fbox{$\, \phantom{-}\, $} \hspace*{1cm}   $T_{\rm T}=2\pi \left({m \over A \rho_{\rm A} g}\right)^{1/2}$
\item[(b)] \fbox{$\, \phantom{-}\, $} \hspace*{1cm}   $T_{\rm T}=2\pi \left({A \rho_{\rm A} g\over m}\right)^{-1}$
\item[(c)] \fbox{$\, \phantom{-}\, $} \hspace*{1cm}   $T_{\rm T}=2\pi \left({A \rho_{\rm A} g\over m}\right)^{1/2}$
\end{enumerate}

%\newpage

%\item  Un muelle de constante el�stica $k=0,98$~N~m$^{-1}$ tiene colgada una masa de $m=1,05$~kg en su extremo libre. Su per�odo de 
%oscilaci�n $T_{\rm k}$ ser� :
%\begin{enumerate}
%\item[(a)] \fbox{$\, \phantom{-}\, $} \hspace*{1cm}     $T_{\rm k} \approx  4$~s 
%\item[(b)] \fbox{$\, \phantom{-}\, $} \hspace*{1cm}   $T_{\rm k}\approx 5$~s
%\item[(c)] \fbox{$\, \phantom{-}\, $} \hspace*{1cm}   $T_{\rm k} \approx 6$~s
%\end{enumerate}

\item  Un muelle de constante el\'astica $k=0,98$~N~m$^{-1}$ tiene colgada una masa de $m=1,05$~kg en su extremo libre. 
Si este oscilador fuera trasladado a la {\bf Luna}, su per\'iodo de oscilaci\'on  $T_{\rm L}$, respecto de su per\'iodo de oscilaci\'on en la Tierra, $T_{\rm T}$,
%oscilaci�n $T_{\rm k}$ ser� :
\begin{enumerate}
\item[(a)] \fbox{$\, \phantom{-}\, $} \hspace*{1cm}   ser\'a menor, $T_{\rm L} <  T_{\rm T}$ 
\item[(b)] \fbox{$\, \phantom{-}\, $} \hspace*{1cm}   ser\'a mayor,   $T_{\rm L} >   T_{\rm T}$
\item[(c)] \fbox{$\, \phantom{-}\, $} \hspace*{1cm}   ser\'a el mismo, $T_{\rm L} =  T_{\rm T}$
\end{enumerate}


%\item  Un buzo de Descartes se encuentra situado en su posici�n de equilibrio debajo de la superficie del agua (profundidad de no retorno). Al ser perturbado:
%\begin{enumerate}
%\item[(a)] \fbox{$\, \phantom{-}\, $} \hspace*{1cm}   oscilar� con frecuencia $T_\rD=2\pi \left(A \rho_{\rm A} g/m\right)^{-1/2}$
%\item[(b)] \fbox{$\, \phantom{-}\, $} \hspace*{1cm}   oscilar� con frecuencia $T_\rD=2\pi \left(A \rho_{\rm A} g/m\right)^{1/2}$
%\item[(c)] \fbox{$\, \phantom{-}\, $} \hspace*{1cm}   no oscilar�, pues el equilibrio es inestable
%\end{enumerate}


%\item  Para un p�ndulo de torsi�n,  de constante de torsi�n $k_\theta$, conectado a un cuerpo  con  momento de inercia $I$, su per�odo de 
%oscilaci�n $T_\theta$ viene dado por:
%\begin{enumerate}
%\item[(a)] \fbox{$\, \phantom{-}\, $} \hspace*{1cm}    $T_\theta=2\pi \left(I/k_\theta\right)^{-1/2}$
%\item[(b)] \fbox{$\, \phantom{-}\, $} \hspace*{1cm}   $T_\theta=2\pi \left(I/k_\theta\right)^{1/2}$
%\item[(c)] \fbox{$\, \phantom{-}\, $} \hspace*{1cm}   $T_\theta=2\pi \left(I/k_\theta\right)^{2}$
%\end{enumerate}

%\item  Un tubo en U contiene mercurio (Hg). Si el nivel del mercurio se perturba, el l�quido oscila. Sea $L$ la longitud del tubo que contiene
%mercurio, $A$ su secci�n  y $\rho_\rH$ la densidad del mercurio. El per�odo de oscilaci�n $T_\rU$ de la columna de mercurio es:
%\begin{enumerate}
%\item[(a)] \fbox{$\, \phantom{-}\, $} \hspace*{1cm}   $T_\rU=2\pi \left(2L/g\right)^{1/2}$
%\item[(b)] \fbox{$\, \phantom{-}\, $} \hspace*{1cm}   $T_\rU=2\pi \left(2A\rho_\rH/Lg\right)^{1/2}$
%\item[(c)] \fbox{$\, \phantom{-}\, $} \hspace*{1cm}   $T_\rU=2\pi \left(2AL\rho_\rH/g\right)^{1/2}$
%\end{enumerate}


\item Un carrito de carril sin rozamiento, unido a dos muelles, se encuentra forzado
con su frecuencia natural y amortiguado mediante una placa de aluminio que sobresale
del carro y que oscila entre dos imanes. {\bf Aproximamos} los dos imanes a la placa
de aluminio y esperamos a que se alcance el estado estacionario. La {\bf frecuencia}
$\omega$ de la oscilaci\'on del carro:
\begin{enumerate}
\item[(a)] \fbox{$\, \phantom{-}\, $} \hspace*{1cm}   ha aumentado
\item[(b)] \fbox{$\, \phantom{-}\, $} \hspace*{1cm}   ha disminuido
\item[(c)] \fbox{$\, \phantom{-}\, $} \hspace*{1cm}   no ha variado
\end{enumerate}

\item Un carrito de carril sin rozamiento, unido a dos muelles, se encuentra forzado
con su frecuencia natural y amortiguado mediante una placa de aluminio que sobresale
del carro y que oscila entre dos imanes. {\bf Alejamos} los dos imanes de la placa
de aluminio, y esperamos a que se alcance el estado estacionario. La {\bf amplitud}
$A$ de la oscilaci\'on del carro:
\begin{enumerate}
\item[(a)] \fbox{$\, \phantom{-}\, $} \hspace*{1cm}   ha aumentado
\item[(b)] \fbox{$\, \phantom{-}\, $} \hspace*{1cm}   ha disminuido
\item[(c)] \fbox{$\, \phantom{-}\, $} \hspace*{1cm}   no ha variado
\end{enumerate}

\item  Una persona oscila con un {\bf columpio},  no sentada sino situada de pie sobre el asiento del mismo.
Para mantener su oscilaci\'on sin amortiguar, debe:
\begin{enumerate}
\item[(a)] \fbox{$\, \phantom{-}\, $} \hspace*{1cm}   descender en la parte baja y ascender en la parte alta
\item[(b)] \fbox{$\, \phantom{-}\, $} \hspace*{1cm}   ascender en la parte baja y descender en la parte alta
\item[(c)] \fbox{$\, \phantom{-}\, $} \hspace*{1cm}   ascender en la parte baja y ascender en la parte alta
\end{enumerate}

\end{enumerate}    
	
\end{document}
