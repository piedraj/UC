\documentclass[11pt]{articulo}
\usepackage{amsmath}
\usepackage{graphicx}
\usepackage[spanish, es-tabla]{babel}
\usepackage{booktabs}
\setlength{\parindent}{0px}
\usepackage{anysize}
\marginsize{2.8cm}{2.8cm}{2.3cm}{2.5cm}

\begin{document}

\title{\bf Estudio de colisiones prot\'on-prot\'on a 13,6 TeV en el experimento CMS}

\author{Alumno: Alain Verduras Schaeidt\\
    Tutor: J\'onatan Piedra G\'omez}
\maketitle

\section{Introducci\'on} 

Tras una larga parada el gran colisionador de hadrones (LHC) reanudar\'a su
actividad en febrero de 2022, chocando protones a una energ\'ia de centro de masas
de 13,6 TeV, energ\'ia nunca antes alcanzada. La idea de este proyecto es entender
y analizar las colisiones que en 2022 detectar\'a el experimento Compact Muon Solenoid
(CMS), experimento en el que participan investigadoras e investigadores del
Departamento de F\'isica Moderna. Concretamente una de las \'areas de trabajo
corresponde a la f\'isica del bos\'on de Higgs, en particular el estudio de la
producci\'on del bos\'on de Higgs junto a un bos\'on W o Z. Esta es la tercera
forma m\'as probable de producir bosones de Higgs en el LHC~\cite{HW}, tras la
fusi\'on de gluones (m\'as probable) y la fusi\'on de bosones vectoriales.
El canal de desintegraci\'on a estudiar ser\'ia aquel en el que el bos\'on
W o Z se desintegra hadr\'onicamente (en quarks) y el bos\'on de Higgs se desintegra
en dos bosones W, que a su vez se desintegran lept\'onicamente. Como en la
desintegraci\'on lept\'onica de los bosones W se producen neutrinos, el canal
elegido tendr\'a en su estado final, adem\'as de jets (formados a partir de los quarks)
y leptones cargados (electrones y/o muones), energ\'ia perdida.

\section{Objetivos generales}

El objetivo principal es medir por primera vez la secci\'on eficaz de producci\'on del bos\'on de Higgs cuando se produce en asociaci\'on con un bos\'on W o Z. Para ello se debe mejorar la separaci\'on entre la se\~nal (producci\'on del bos\'on de Higgs en asociaci\'on con un bos\'on W o Z) y los fondos, tanto los que corresponden a otros procesos de producci\'on del bos\'on de Higgs, como fondos bien conocidos del modelo est\'andar (ME) entre los que destacan la producci\'on de bosones Z, bosones W y quarks top.

\section{Objetivos espec\'ificos}

\begin{itemize}

\item{Conocimientos te\'oricos: modelo est\'andar, f\'isica de part\'iculas, colisiones de protones, f\'isica del bos\'on de Higgs, aceleradores de part\'iculas y detectores de colisiones.}

\item{Conocimiento detallado de los procesos de estudio dados en las colisiones prot\'on-prot\'on (se\~nal) y de los procesos con productos finales similares a los procesos de se\~nal pero que no son de inter\'es para este estudio (fondo).}

\item{Conocimiento detallado del proceso de an\'alisis, desde el dise\~no del detector (opcional) hasta la toma de datos, {\it trigger} de los mismos, reconstrucci\'on, producci\'on de objetos elaborados (muones, electrones, jets, b-jets, energ\'ia perdida) y producci\'on de simulaciones de Monte Carlo (MC).}

\item{Dise\~no de cortes secuenciales para reducir la cantidad de sucesos de fondo respecto a los de se\~nal.}

\item{Conocimiento de m\'etodos de an\'alisis basados en redes neuronales, llegando a {\it deep neural networks}.}

\item{Dado el elevado n\'umero de colisiones (reales y simuladas) a analizar, manejo de {\it big data}.}

\item{Caracterizaci\'on de errores sistem\'aticos.}

\item{Manejo del c\'odigo dise\~nado por la colaboraci\'on CMS en los lenguajes de programaci\'on C++ y Python para el an\'alisis de datos de CMS.}

\item{Realizaci\'on de un informe final de car\'acter cient\'ifico que recoja el trabajo realizado en colaboraci\'on con el Departamento de F\'isica Moderna y las conclusiones del mismo.}

\item{Participaci\'on en una colaboraci\'on cient\'ifica internacional.}

\item{Participaci\'on en otras posibles tareas de formaci\'on, asesoramiento o investigaci\'on en colaboraci\'on con el Departamento de F\'isica Moderna.}

\item{Desarrollo de competencias espec\'ificas y generales propias del Grado en F\'isica.}

\end{itemize}

\section{Competencias generales}

\begin{itemize}

\item{{\bf Aplicaci\'on.} Que el estudiante sepa aplicar sus conocimientos a su trabajo o vocaci\'on de una forma profesional y posea las competencias que suelen demostrarse por medio de la elaboraci\'on y defensa de argumentos y la resoluci\'on de problemas dentro de su \'area de estudio.}

\item{{\bf An\'alisis.} Que el estudiante tenga la capacidad de reunir e interpretar datos relevantes para emitir juicios que incluyan una reflexi\'on sobre temas relevantes de \'indole social, cient\'ifica o \'etica.}

\item{{\bf Comunicaci\'on.} Que el estudiante pueda transmitir informaci\'on, ideas, problemas y soluciones a un p\'ublico tanto especializado como no especializado.}

\item{{\bf Aprendizaje.} Que el estudiante haya desarrollado aquellas habilidades de aprendizaje necesarias para emprender estudios posteriores con un alto grado de autonom\'ia.}

\end{itemize}

\section{Competencias espec\'ificas}

\begin{itemize}

\item{{\bf Conocimiento.} Conocer y comprender los fen\'omenos f\'isicos, las teor\'ias, leyes y modelos que los rigen, incluyendo su dominio de aplicaci\'on y su formulaci\'on en lenguaje matem\'atico.}

\item{{\bf Aplicaci\'on.} Saber utilizar los m\'etodos matem\'aticos, anal\'iticos y num\'ericos b\'asicos, para la descripci\'on del mundo f\'isico, incluyendo en particular la elaboraci\'on de teor\'ias y modelos y el planteamiento de medidas experimentales.}

\item{{\bf Comunicaci\'on.} Saber presentar de forma adecuada, en castellano y en su caso en ingl\'es, el estudio realizado de un problema f\'isico, comenzando por la descripci\'on del modelo utilizado e incluyendo los detalles matem\'aticos, num\'ericos e instrumentales y las referencias pertinentes a otros estudios.}

\item{{\bf Aprendizaje.} Saber acceder a la informaci\'on necesaria para abordar un trabajo o estudio utilizando las fuentes adecuadas, incluyendo literatura cient\'ifico-t\'ecnica en ingl\'es, y otros recursos digitales.}

\item{{\bf Herramientas.} Dominar el uso de las t\'ecnicas de computaci\'on necesarias en la aplicaci\'on de los modelos. Conocer los principios y t\'ecnicas de medida as\'i como la instrumentaci\'on m\'as relevante en los diferentes campos de la f\'isica de part\'iculas.}

\item{{\bf Iniciativa.} Ser capaz de trabajar de modo aut\'onomo, mostrando iniciativa propia y sabiendo organizarse para cumplir los plazos marcados. Aprender a trabajar en equipo, contribuyendo constructivamente y asumiendo responsabilidades y liderazgo.}

\end{itemize}

\section{Tareas a realizar} 

El conjunto de tareas a realizar parte de la adqusici\'on de los conocimientos b\'asicos y necesarios de la f\'isica de part\'iculas. Desde la teor\'ia que predice y describe los acontecimientos tras una colisi\'on prot\'on-prot\'on hasta el an\'alisis de dichas colisiones. El estudiante se integrar\'a en el grupo de f\'isica de part\'iculas del IFCA, asistiendo a las reuniones semanales tanto del grupo como dentro de la Colaboraci\'on CMS. Su trabajo consistir\'a en medir la secci\'on eficaz de producci\'on de bosones de Higgs cuando son producidos junto a un bos\'on W o Z.

\section{Aspectos a destacar} 

\begin{itemize}

\item{{\bf Contenido innovador.} En febrero de 2022 el LHC reanuda su actividad colisionando protones a una energ\'ia de centro de masas de 13,6~TeV~\cite{HL}. Actualmente la m\'axima energ\'ia de centro de masas que se ha conseguido, en el propio LHC, es de 13 TeV. Aumentar la energ\'ia de colisi\'on permitir\'a que la secci\'on eficaz de producci\'on de part\'iculas pesadas como el bos\'on de Higgs aumente. Esto permitir\'a realizar medidas m\'as precisas de las propiedades de estas part\'iculas para as\'i compararlas con las predicciones te\'oricas. Adem\'as, poder medir regiones m\'as energ\'eticas permitir\'a utilizar las t\'ecnicas de an\'alisis de colas (zonas de las distribuciones estad\'isticamente menos pobladas) con mayor precisi\'on. Recientemente se han observado indicios de nueva f\'isica en experimentos como el Muon g-2 de Fermilab~\cite{g2} y esta mejora del dispositivo experimental del LHC permitir\'a experimentar en regiones que nunca se ha podido experimentar antes.}

\item{{\bf Aplicaci\'on del trabajo a desarrollar dentro del departamento.} Este proyecto se engloba dentro de los objetivos del grupo de f\'isica de part\'iculas del IFCA en el experimento CMS. La participaci\'on del alumno ser\'a de gran utilidad para conocer mejor la f\'sica del bos\'on de Higgs.}

\item{{\bf Continuidad en futuros proyectos de investigaci\'on.} Este proyecto se centra en el aumento de energ\'ia de centro de masas que tendr\'a el LHC en 2022. Este aumento de energ\'ia, que se espera que llegue hasta los 14 TeV, es parte de la preparaci\'on del proyecto High Luminosity LHC (HL-LHC)~\cite{HL} en el que trabaja el CERN y que se espera est\'e listo para 2027. El HL-LHC consiste en aumentar en un factor 10 la luminosidad del LHC, es decir, tener mayor numero de colisiones de protones por segundo y por tanto una mayor estad\'istica a la hora de realizar los an\'alisis. Este proyecto de colaboraci\'on podr\'ia por tanto sentar las bases para futuros proyectos de colaboraci\'on relacionados con el HL-LHC, tanto de preparaci\'on para la toma de datos como del posterior an\'alisis de estos, dado que el IFCA est\'a estrechamente relacionado con este proyecto~\cite{IFCA0}. A m\'as largo plazo el IFCA est\'a relacionado con el dise\~no de detectores para el posible futuro colisionador lineal internacional ILC~\cite{IFCA}. El ILC es un proyecto de acelerador lineal de leptones, en el que colabora la comunidad europea de f\'isica de part\'iculas, que se espera que funcione como una f\'abrica de bosones de Higgs~\cite{ILC}.}

\item{{\bf Introducci\'on de nuevas tecnolog\'ias en los m\'etodos de trabajo.} En primer lugar, para trabajar con los datos tomados por CMS se debe utilizar el c\'odigo desarrollado por el grupo de CMS en C++ y Python. Adem\'as de estos dos lenguajes de programaci\'on tambi\'en es necesaria la herramienta de an\'alisis cient\'ifico ROOT~\cite{Root} desarrollada por el CERN para poder manejar cantidades tan grandes de datos. Tambi\'en se trabajar\'a con simulaciones de MC basadas en el modelo est\'andar y/o en nuevos modelos te\'oricos. Ser\'a necesaria la comprensi\'on de estas simulaciones y su manejo. Para el an\'alisis se utilizar\'an t\'ecnicas de cortes secuenciales y t\'ecnicas de computaci\'on m\'as sofisticadas como el {\it machine learning}~\cite{ML}. Se har\'a uso de redes neuronales y por tanto ser\'a necesario el conocimiento, manejo de estas, la optimizaci\'on del n\'umero de capas, de neuronas por capa y de las funciones de cada neurona de la red para obtener un mejor an\'alisis de los datos. Finalmente, para poder hacer frente al manejo de tantos datos y de procesos de computaci\'on con elevados consumos de CPU se trabajar\'a en el entorno grid dado que muchos de estos procesos son paralelizables.}

\end{itemize}

\begin{thebibliography}{X}
	
\bibitem{HW} CMS Collaboration. The CMS experiment at the CERN LHC, 2016. Measurements of properties of the Higgs boson decaying to a W boson pair in pp collisions at $\sqrt{s} = 13~{\rm TeV}$, 2019. Physics Letters B, Volume 791, Pages 96-129. https://doi.org/10.1016/j.physletb.2018.12.073.
	
\bibitem{HL} HL-LHC: Project Schedule Jan. 12, 2012. [Online]. Available: https://project-hl-lhc-industry.web.cern.ch/content/project-schedule.

\bibitem{g2} B. Abi et al. [Muon g-2], Phys. Rev. Lett.\textbf{126}, 141801 (2021) doi: 10.1103/PhysRevLett.126.141801[arXiv:2104.03281 [hep-ex]].
  
\bibitem{IFCA0} Instituto de F\'isica de Cantabria (30 de junio de 2020). \textit{New precision timing detector for the CMS HL-LHC upgrade}. https://ifca.unican.es/en-us/news/Paginas/New-precision-timing-detector-for-the-CMS-HL-LHC-upgrade.aspx.

\bibitem{IFCA} Instituto de F\'isica de Cantabria (20 de junio de 2020). \textit{El CERN actualiza la estrategia para el futuro de la f\'isica de part\'iculas en Europa}. https://ifca.unican.es/es-es/news/Paginas/EL-CERN-actualiza-la-estrategia-para-el-futuro-de-la-fisica-de-particulas-en-Europa.aspx.
  
\bibitem{ILC} The European Strategy Group. Deliberation document on the 2020 Update of the European Strategy for Particle Physics, Geneva, CERN-ESU-014, 2020, http://cds.cern.ch/record/2720131, doi: 10.17181/ESU2020Deliberation.
    
\bibitem{Root}About ROOT. https://root.cern.ch/about-root.
  
\bibitem{ML} Oropeza-Barrera, Cristina. 2020. Uso de t\'ecnicas de machine learning en el experimento CMS. FIGURAS REVISTA ACAD\'EMICA DE INVESTIGACI\'ON 1 (3). https://doi.org/10.22201/fesa.figuras.2020.1.3.121.
	
\end{thebibliography}
\end{document}
