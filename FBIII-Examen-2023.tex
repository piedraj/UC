\documentclass[11pt]{articulo}
\usepackage{multirow}

\textwidth 16.5cm 
\textheight 25.5cm
\topmargin -2cm
\oddsidemargin 0cm   
\evensidemargin 0cm  
\baselineskip0.6cm plus 1pt minus 1pt


\begin{document}
\begin{verse}
{\Large UNIVERSIDAD DE CANTABRIA}\\ 
\vspace*{0.5cm}
{\normalsize \rm Curso 1 de Grado en F\'isica y Doble Grado en F\'isica y Matem\'aticas}\\
{\normalsize \rm F\'isica b\'asica experimental III: la materia y sus propiedades}\\ 
{\normalsize \rm Examen de los temas 8, 9 y 10}\\
{\normalsize \rm 22 de mayo de 2023}\\
\end{verse} 

\vspace*{0.25cm}

NOMBRE Y APELLIDOS:\\

\vspace*{0.25cm}
 
\hrule
\begin{enumerate}
\item[] El tiempo disponible es de 30 minutos.
\item[] Todas las cuestiones y ejercicios punt\'uan igual.
\item[] La extensi\'on m\'axima es de dos caras.
\end{enumerate}
\hrule
          
\begin{enumerate}

\item Contestar brevemente a {\bf\underline{dos de las tres}} siguientes cuestiones.

\begin{enumerate}
\item Explica las caracter\'isticas y las diferencias principales entre los rayos alfa, beta y gamma.
\item Explica qu\'e es y c\'omo apareci\'o el neutrino por primera vez.
\item Explica los or\'igenes te\'orico y experimental de la antimateria, e indica al menos dos lugares donde podemos encontrar antimateria.
\end{enumerate}

\item Resolver {\bf\underline{dos de los tres}} siguientes ejercicios, justificando los resultados obtenidos.

\begin{enumerate}

\item Suponiendo que el n\'ucleo hijo tiene una energ\'ia cin\'etica despreciable, determina la energ\'ia m\'axima (en MeV) del electr\'on emitido en la desintegraci\'on beta del ${\rm _{~6}^{14}C}$, siendo ${\rm 14,003242~u}$ la masa at\'omica del ${\rm _{~6}^{14}C}$ y ${\rm 14,003074~u}$ la masa at\'omica del ${\rm _{~7}^{14}N}$.

\item Calcular la energ\'ia (en MeV) que debe tener una part\'icula alfa para alcanzar la superficie de un n\'ucleo de oro (${\rm _{~79}^{197}Au}$) suponiendo que el n\'ucleo de oro permanece inm\'ovil. (${\rm r_0 = 1,2~fm}$)

\item En una roca terrestre se observa un contenido de $10^9$~\'atomos/g de ${\rm _{38}^{87}Sr}$ y $2 \times 10^{10}$~\'atomos/g de ${\rm _{37}^{87}Rb}$. Asumiendo que el estroncio procede de la desintegraci\'on beta del rubidio, cuyo per\'iodo de semidesintegraci\'on es de $4,88 \times 10^{10}$~a\~{n}os, calcula la edad de la roca (en a\~{n}os).

\end{enumerate}
\end{enumerate}

%%%%%%%%%%%%%%%%%%%%%%%%

\vspace*{0.25cm}

%%%\begin{table}[h!]
%%%\begin{center}
\begingroup
\renewcommand{\arraystretch}{1.5} % Default value: 1
\begin{tabular}{lll}
Magnitud & S\'imbolo & Valor\\
\hline
Constante de Coulomb & $k_e$ & ${\rm 8,987\,551\,787 \times 10^9~N \cdot m^2 / C^2}$\\
\multirow{2}{*}{Masa del electr\'on} & \multirow{2}{*}{$m_e$} & ${\rm 5,485\,799\,03 \times 10^{-4}~u}$\\
                                     &                        & ${\rm 0,510\,999\,06~MeV/}c^2$\\
Electr\'on-voltio    & eV    & ${\rm 1,602\,177\,33 \times 10^{-19}~J}$\\
Carga elemental      & $e$   & ${\rm 1,602\,177\,33 \times 10^{-19}~C}$\\
\hline
\multicolumn{3}{l}{Algunas constantes fundamentales}\\
\end{tabular}
\endgroup
%%%\end{center}
%%%\end{table}

\end{document}
