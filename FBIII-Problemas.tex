\documentclass[11pt]{articulo}
\usepackage{multirow}
\usepackage{amsmath}

\textwidth 16.5cm 
\textheight 25.5cm
\topmargin -2cm
\oddsidemargin 0cm   
\evensidemargin 0cm  
\baselineskip0.6cm plus 1pt minus 1pt


\begin{document}
\begin{verse}
{\Large UNIVERSIDAD DE CANTABRIA}\\ 
\vspace*{0.5cm}
{\normalsize \rm Curso 1 de Grado en F\'isica y Doble Grado en F\'isica y Matem\'aticas}\\
{\normalsize \rm F\'isica b\'asica experimental III: la materia y sus propiedades}\\ 
{\normalsize \rm Problemas de los temas 8 y 9}\\
{\normalsize \rm 16 de mayo de 2025}\\
\end{verse} 

\vspace*{0.25cm}

\begin{enumerate}

\item Suponiendo que el peso de una persona es de $50~{\rm kg}$ estima el orden de magnitud de (a) su n\'umero de protones, (b) su n\'umero de neutrones, y (c) su n\'umero de electrones.

\item Realizamos un experimento en el que lanzamos part\'iculas $\alpha$ de $24,808~{\rm MeV}$ que alcanzan la superficie del n\'ucleo de un elemento. Indica de qu\'e elemento se trata suponiendo que tiene el mismo n\'umero de protones que de neutrones.

\item Una estrella cuya masa es el doble que la del Sol ($M_{\rm Sol} = 1,99 \times 10^{30}~{\rm kg}$) se colapsa combinando sus protones y sus electrones para formar una estrella de neutrones. Si pensamos en esta estrella como en un gigantesco n\'ucleo at\'omico, ?`qu\'e radio tendr\'a (en kil\'ometros)?

\item Calcular (en ${\rm MeV}$) la diferencia en las energ\'ias de enlace de $_{~8}^{15}{\rm O}$ y $_{~7}^{15}{\rm N}$.

\item Una muestra de material radioactivo contiene $10^{15}$~\'atomos y tiene una actividad de $6\times 10^{11}~{\rm Bq}$. Indica (en minutos) su per\'iodo de semidesintegraci\'on.

\item Una muestra reci\'en preparada de un is\'otopo radioactivo tiene una actividad de $10~{\rm mCi}$. Despu\'es de 4 horas su actividad es de $8~{\rm mCi}$. (a) Hallar la constante de desintegraci\'on (en unidades del Sistema Internacional) y el per\'iodo de semidesintegraci\'on (en horas). (b) ?`Cu\'antos \'atomos del is\'otopo hab\'ia en la muestra reci\'en preparada? (c) ?`Cu\'al es la actividad (en ${\rm mCi}$) de la muestra 30 horas despu\'es de ser preparada?

\item El per\'iodo de semidesintegraci\'on de un n\'ucleo radioactivo es $T_{1/2}$ y tenemos una muestra con una actividad inicial $R_0$. Calcular anal\'iticamente el n\'umero de n\'ucleos que se desintegran en el intervalo comprendido entre los tiempos $T_{1/2}$ y $2 \,T_{1/2}$.

\item Calcula (en ${\rm MeV}$) la energ\'ia liberada en la siguiente desintegraci\'on.
%
\begin{equation*}
{\rm {_{~92}^{238}U} \longrightarrow {_{~90}^{234}Th} + {_2^4He}}
\end{equation*}

\item El n\'ucleo ox\'igeno-15 se desintegra por captura electr\'onica en nitr\'ogeno-15 y un neutrino. (a) Escribe la reacci\'on nuclear. (b) Escribe la reacci\'on sufrida por una part\'icula en el n\'ucleo. (c) Determina la energ\'ia del neutrino (en MeV) ignorando el retroceso del nitr\'ogeno-15. 

\item El oro solo tiene un is\'otopo natural $_{~79}^{197}{\rm Au}$. Si irradiamos oro natural con un flujo de neutrones lentos se emiten electrones. (a) Escribir la ecuaci\'on de la reacci\'on. (b) Calcular (en MeV) la energ\'ia m\'axima de los electrones emitidos.

\item (a) Hallar la energ\'ia liberada (en MeV) en la siguiente reacci\'on de fisi\'on. (b) Calcula el porcentaje de la masa inicial que se transforma.
%
\begin{equation*}
{\rm n + {_{~92}^{235}U} \longrightarrow {_{~56}^{141}Ba} + {_{36}^{92}Kr} + 3n}
\end{equation*}

\item Se ha calculado que existen aproximadamente $10^9$ toneladas de uranio natural, del cual el $0,7\%$ es el is\'otopo fisionable $^{235}{\rm U}$. Supongamos que todas las necesidades energ\'eticas mundiales ($7 \times 10^{12}~{\rm J/s}$) son cubiertas con la fisi\'on de $^{235}{\rm U}$ en reactores nucleares, liberando 208~MeV por reacci\'on. ?`Cu\'anto tiempo (en a\~nos) durar\'ia este suministro?

\end{enumerate}

%%%%%%%%%%%%%%%%%%%%%%%%

\vspace*{0.25cm}

\begingroup
\renewcommand{\arraystretch}{1.5} % Default value: 1
\begin{tabular}{lll}
\multicolumn{3}{l}{Algunas constantes fundamentales}\\
\hline
Magnitud & S\'imbolo & Valor\\
\hline
N\'umero de Avogadro     & $N_{\rm A}$ & ${\rm 6,022\,136\,7 \times 10^{23}}$~part\'iculas/mol\\
Constante de Coulomb     & $k_e$       & ${\rm 8,987\,551\,787 \times 10^9~N \cdot m^2 / C^2}$\\
Electr\'on-voltio        & eV          & ${\rm 1,602\,177\,33 \times 10^{-19}~J}$\\
Carga elemental          & $e$         & ${\rm 1,602\,177\,33 \times 10^{-19}~C}$\\
Curie                    & Ci          & ${\rm 3,7 \times 10^{10}~Bq}$\\
Unidad de masa at\'omica & u           & $931,494\,32~{\rm MeV}/c^2$\\
\hline
\end{tabular}
\endgroup

\vspace*{0.5cm}

\begingroup
\renewcommand{\arraystretch}{1.5} % Default value: 1
\begin{tabular}{lr}
\multicolumn{2}{l}{Algunas masas at\'omicas}\\
\hline
Elemento o part\'icula & Masa at\'omica (u)\\
\hline
electr\'on             & $  0,000\,549$\\
neutr\'on              & $  1,008\,665$\\
${\rm _{1}^{1}H}$      & $  1,007\,825$\\
${\rm _{2}^{4}He}$     & $  4,002\,602$\\
${\rm _{~7}^{15}N}$    & $ 15,000\,108$\\
${\rm _{~8}^{15}O}$    & $ 15,003\,065$\\
${\rm _{36}^{92}Kr}$   & $ 91,897\,3$\\
${\rm _{~56}^{141}Ba}$ & $140,913\,9$\\
${\rm _{~79}^{197}Au}$ & $196,966\,543$\\
${\rm _{~80}^{198}Hg}$ & $197,966\,743$\\
${\rm _{~90}^{234}Th}$ & $234,043\,593$\\
${\rm _{~92}^{235}U}$  & $235,043\,915$\\
${\rm _{~92}^{238}U}$  & $238,050\,784$\\
\hline
\end{tabular}
\endgroup

\end{document}
