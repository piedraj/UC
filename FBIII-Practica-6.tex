\documentclass[11pt]{articulo}
\usepackage{multirow}
\usepackage{amsmath}

\textwidth 16.5cm 
\textheight 25.5cm
\topmargin -2cm
\oddsidemargin 0cm   
\evensidemargin 0cm  
\baselineskip0.6cm plus 1pt minus 1pt

\begin{document}
\begin{verse}
{\Large UNIVERSIDAD DE CANTABRIA}\\ 
\vspace*{0.5cm}
{\normalsize \rm Curso 1 de Grado en F\'isica y Doble Grado en F\'isica y Matem\'aticas}\\
{\normalsize \rm F\'isica b\'asica experimental III: la materia y sus propiedades}\\ 
{\normalsize \rm Pr\'actica 6}\\
\end{verse} 

\vspace*{0.25cm}

\begin{enumerate}

\item {\bf Informe}

\begin{itemize}

\item M\'aximo 4 p\'aginas + Bibliograf\'ia.

\item {\bf (0.5)} M\'aximo tres erratas.

\end{itemize}

\item {\bf T\'itulo}

\begin{itemize}

\item {\bf (0.25)} Debe contener las palabras colisi\'on, prot\'on y CMS.

\end{itemize}

\item {\bf Introducci\'on}

\begin{itemize}

\item M\'aximo 1/3 p\'agina.

\item {\bf (0.25)} Hablar sobre el CERN, LHC y CMS.

\end{itemize}

\item {\bf Objetivos y f\'isica}

\begin{itemize}

\item M\'aximo 1/3 p\'agina.

\item {\bf (0.5)} Hablar de la f\'isica de choques prot\'on-prot\'on y de la producci\'on de ${\rm W}$, ${\rm Z}$ y Higgs.

\end{itemize}

\item {\bf Herramientas}

\begin{itemize}

\item M\'aximo 1/3 p\'agina.

\end{itemize}

\item {\bf Resultados}

\begin{itemize}

\item M\'aximo 2 p\'aginas.

\item Tabla de 100 sucesos a doble columna. No debe estar partida y ocupar\'a como m\'aximo una de las dos p\'aginas de resultados.

\item Tabla que contenga las fracciones ${\rm W/Z}$, ${\rm W^+/W^-}$ y ${\rm e/\mu}$.

\item Histograma de masa de dos leptones.

\item Histograma de masa de cuatro leptones.

\item {\bf (0.5)} Tener en cuenta los resultados de todo el grupo.

\item {\bf (0.5)} Aunque una colisi\'on sea {\it zoo} hay que reflejar su topolog\'ia.

\end{itemize}

\item {\bf An\'alisis y discusi\'on de los resultados}

\begin{itemize}

\item M\'aximo 1/2 p\'agina.

\item {\bf (0.5)} Razonar y buscar los valores esperados para ${\rm W/Z}$, ${\rm W^+/W^-}$ y ${\rm e/\mu}$.

\item {\bf (0.5)} Localizar las resonancias de baja masa invariante. La masa del $J/\psi$ son ${\rm 3.1~GeV}$ y la masa del $\Upsilon(1S)$ son ${\rm 9.5~GeV}$.

\end{itemize}

\item {\bf Conclusiones}

\begin{itemize}

\item M\'aximo 1/2 p\'agina.

\item {\bf (0.5)} Hablar de la estad\'istica y de otras posibles desintegraciones no consideradas.

\end{itemize}

\item {\bf Bibliograf\'ia}

\begin{itemize}

\item {\bf (0.5)} Tener una bibliograf\'ia m\'inima.

\item {\bf (0.25)} Llamar a las referencias en el texto principal.

\end{itemize}

\end{enumerate}

\end{document}
