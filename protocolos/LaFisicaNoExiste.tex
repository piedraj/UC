\documentclass[11pt]{articulo}
\usepackage{graphicx}

\baselineskip 25pt
\textwidth 13cm
\textheight 20cm
\topmargin -1.5cm
\oddsidemargin 2cm 
\evensidemargin 2cm 

\input option_keys
\input Definir(99) 
 
\begin{document}

\title{\bf La F�sica no existe.\\ Problemas conceptuales en F�sica} 
 \author{ {\bf J \ G��mez}\\
{\it Aula de la Ciencia}\\ Universidad de Cantabria\\
}
\maketitle

\begin{abstract}
Todos los a�os se publican en las revistas pedag�gicas de F�sica, aquellas que van dirigidas a profesores de bachillerato y universidad, 
art�culos relativos a temas que podr�a pensarse que ya est�n bien entendidos.  Sobre algunos de ellos, la pol�mica contin�a.
\end{abstract}


\begin{enumerate}
\item El extra�o mundo de la F�sica Cl�sica.
\item Fuerzas. Se considera que existen cuatro fuerzas o interacciones fundamentales. Pero la fuerza de contacto es tambi�n abundante. 
\item Fuerzas de Rozamiento
\item Principio de Inercia
\item La Segunda Ley de Newton. Cuerpos extensos.
\item Destrucci�n de energ�a mec�nica.
\item Creaci�n de energ�a mec�nica.
\item Primer Principio de la Termodin�mica. Procesos reversibles.
\item Segundo Principio de la Termodin�mica. Procesos espont�neos.
\item Lagrangianos y hamiltonianos disipativos. 
\item Rotaci�n 
\item Principio de Relatividad. 
\item Constancia de la velocidad de la luz. Efecto Cerenkov
\item El parad�jico mundo de la F�sica Relativista
\item Relatividad Especial y Tercera Ley de Newton
\item Contracci�n de Lorentz. 
\item Ecuaci�n de Einstein
\item Termodin�mica Relativista
\item Electrodin�mica y Relatividad. La carga y el cable
\item Generador Van de Graaff. 
\item Rotaci�n y Relatividad.
\end{enumerate}






 

 
\end{document}










