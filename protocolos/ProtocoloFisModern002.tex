\documentclass[11pt]{articulo}
\usepackage{graphicx}

\baselineskip 25pt
\textwidth 13cm
\textheight 20cm
\topmargin -1.5cm
\oddsidemargin 2cm 
\evensidemargin 2cm 

\input option_keys
\input Definir(99) 
 
\begin{document}

\title{\bf Aula `Espacio Tocar la Ciencia'} 
 \author{ {\bf J \ G��mez}\\
{\it Aula de la Ciencia}\\ Universidad de Cantabria\\
}
\maketitle


\noindent
{\huge \bf Protocolo de Experiencias\\ de  F�sica Moderna}
\vspace*{0.5cm}

\large 

\section{La luz como energ�a}

\begin{enumerate}
\item {\it La temperatura de una vela.}  La temperatura de una vela de color rojizo-anaranjada, es de unos 800 $^\circ$C. Explicaci�n del color. El color es producido por las peque�as part�culas de carbono a esas temperaturas que no se ha quemado por la mezcla deficiente de la parafina evaporada y el ox�geno del aire.  
\item {\it Temperatura de un mechero de alcohol.} Alcanza temperaturas m�s altas que una vela y puede fundir algunos tipos de vidrio. 
\item {\it Temperatura de un mechero Bunsen de butano}. En este mechero, el gas a alta presi�n que sale de la bombona pasa por un estrechamiento donde su velocidad aumenta mucho. En ese estrechamiento hay varios orificios por los que puede entrar el aire --Efecto Venturi--. As�, la mezcla de combustible y aire es muy eficiente y la llama apenas deja residuos de carbono. 
Su temperatura es de unos 1200 $^\circ$C y ya puede fundir vidrio. 
\item {\it Temperatura de un filamento de hierro.} Un filamento de hierro por el que circula una corriente emite luz cuyo color va siendo m�s blanco a medida que aumenta su temperatura  --Ley de Wien--.
\item {\it Temperatura de una barra de grafito.} Una barra de grafito puede alcanzar los 4000 K, lo que significa que emite una luz muy blanca, muy semejante a la del Sol. 

[Nota Hist�rica. Las primeras bombillas de Edison fueron hechas de bamb� carbonizado. Luego pasaron a ser hechas de wolframio o tungsteno, el metal con el punto de fusi�n m�s alto. El tungsteno fue descubierto por los hermanos  Jos� y Fausto Elhuyar, en 1883 pero durante mucho tiempo se atribuy� al sueco  Carl Wilhelm Scheele --tungsteno significa 'piedra negra' en sueco. 
No hay ning�n elemento de la Tabla Peri�dica que haga referencia a Espa�a.]

\item {\it Temperatura del Sol.} La temperatura en la corona solar es de unos 6000 K. El m�ximo se encuentra, de acuerdo con la ley de Wien, en el color verde.  Quemar un papel, pintado, con la luz del Sol. �Por qu� son verdes las plantas?
\item {\it Placa fotoel�ctrica.} Efecto fotoel�ctrico. Mover un motor el�ctrico con la luz.

\item {\it Placa fosforescente t diodos emisores de luz de diferentes colores.} Se puede recrear el efecto fotoel�ctrico, como efecto cu�ntico, 
utilizando diodos emisores de luz de diferentes colores, del rojo al ultravioleta, utilizando una placa fosforescente. Para los colores entre rojo y verde, no hay efecto sobre la placa, que empieza a emitir luz fosforescente a partir del verde.  El efecto cu�ntico proviene de que los electrones no saltan de un estado a otro hasta que no les llega un cuanto de luz con energ�a suficiente.

\item Ruptura diel�ctrica del aire. M�quina de Wimshurst.
\item Transformador. Escalera de Jacob.
\item Chispas de color utilizando una bobina de Tesla.
\item Espectros de las sales a la llama. Estroncio rojo, cobre verde, sodio naranja, etc.
\item L�mpara de plasma. 
\item Espectros de gases enrarecidos.  
\item Redes de difracci�n y prismas.
\item Rayos cat�dicos. Aparato cruz de Malta.
\item Medida de la relaci�n carga-masa del electr�n. Aparato de Thomson.
\item Efectos ondulatorios en electrones. Aparato de la dualidad onda-corp�sculo en electrones.
\item {\it Fluorescencia.} Emiten al ser iluminados con luz ultravioleta. 
\item {\it Fosforescencia.} Emiten despu�s de ser iluminados con luz ultravioleta. 

\end{enumerate}


 
\end{document}










