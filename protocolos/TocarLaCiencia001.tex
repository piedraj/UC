\documentclass[11pt]{articulo}
\usepackage{amsmath}
\usepackage{graphicx}

\baselineskip 25pt
\textwidth 13cm
\textheight 20cm
\topmargin -1.5cm
\oddsidemargin 2cm 
\evensidemargin 2cm 

%\input option_keys
%\input Definir(99) 
 
\begin{document}

\title{\bf Aula `Espacio Tocar la Ciencia'} 
 \author{ {\bf J.~Guemez}\\
{\it Aula de la Ciencia}\\ Universidad de Cantabria\\
}
\maketitle

{\huge \bf Protocolo de Experiencias de Mec\'anica}
\vspace*{0.5cm}

\large 

La mec\'anica tiene que ver con: (i) cuerpos que se desplazan (cochecito de muelle y coche con globo); (ii) cuerpos que giran (peonza, tip-top); (iii) cuerpos que oscilan (muelles, p\'endulos), (iv) comportamientos extra\~~nos (resorte de caucho, rattleback, resorte de metal). 

Dos ideas principales: (i) ser capaces de hacer estimaciones, es decir, de hacer c\'alculos num\'ericos aproximados, (ii) insistir en la capacidad de predicci\'on de la ciencia, es decir, antes de hacer medidas, anticipar los resultados. Estas habilidades son fundamentales en el dise\~no de experimentos.

\section{Primera Ley de Newton. Principio de Inercia}

\begin{enumerate}
\item {\it Vaso sobre mantel.} Si se coloca un vaso sobre un mantel y el mantel se retira con rapidez, el vaso no se mueve.  Con m\'as masa es m\'as estable, pero hay que retirar el mantel m\'as r\'apidamente.
\item {\it Moneda y cartulina sobre boca de probeta.} Se coloca una moneda sobre una cartulina y todo ello sobre la boca de una probeta. Si se golpea la cartulina con fuerza y en horizontal, la moneda caer\'a al fondo de la probeta. (Opcional)
\item {\it Bola en carril met\'alico.} Sobre un carril met\'alico con forma circular truncada, se lanza una bola. Al abandonar el carril, la bola seguir\'a la trayectoria indicada por el Principio de Inercia (Opcional)
\item {\it Bola en tubo de pl\'astico enrollado.} Una bola gira, empujada por el soplido, en un tubo que gira varias vueltas.  Dar tres opciones al movimiento final de la bola.
\item {\it Distinguir un huevo cocido de un huevo crudo}
  \begin{itemize}
  \item Un huevo cocido y un huevo crudo. (i) Hacer girar r\'apidamente cada uno de los huevos: el cocido se eleva y se pone a girar sobre su eje mayor; el cocido gira con dificultad; (ii) se hacen girar lentamente cada uno de los huevos y con un dedo se le para; el cocido se para definitivamente y el crudo sigue girando despu\'es de levantar el dedo. 
  \item Una peonza Tip-top (que semeja un huevo cocido
  \end{itemize}
\item {\it Experimento con peonza de resorte.} Se utiliza una peonza movida por un resorte controlable (energ\'ia el\'astica controlable) y un reloj digital, para medir el tiempo que tarda la misma peonza y en las mismas condiciones iniciales, en detenerse por completo cuando rueda sobre diferentes superficies: (i) fieltro, (ii) cart\'on, (iii) papel, (iv) metal, (v) vidrio. El mayor tiempo en cada caso debe indicar que en ausencia de rozamiento el tiempo de rotaci\'on deber\'ia ser infinito.
\item {\it Mesa de aire.} La experiencia cotidiana es que aquellos cuerpos que se desplazan, giran u oscilan, terminan por cesar en su movimiento. Se debe al rozamiento. Como cient\'ificos tenemos que aportar pruebas de lo que afirmamos. La mesa de aire se construye para mostrar que si se elimina el rozamiento, es m\'as plausible la Ley de la Inercia. 
\item {\it Bola magn\'etica flotante.} Dotada de un sensor de campo magn\'etico y de un microprocesador, controla 16.000 veces por segundo la posici\'on de la bola, retroaliment\'andose para evitar que caiga.
\item {\it Peonza perpetua.} Peonza que gira sin cesar. (Opcional). Tambi\'en con microprocesador
\end{enumerate}

\begin{enumerate}
\item V\'ideo de c\'amara de alta velocidad de agua en globo de caucho que se pincha el globo y el agua se queda en forma esf\'erica. (Opcional)
\end{enumerate}
 
\section{Segunda Ley de Newton}

La Segunda Ley de Newton $\sum_j {\bf F}_j = m {\bf a}$ resulta bastante dif\'icil de comprobar directamente. 

\begin{description}
\item[Fuerza gravitatoria.] Ca\'ida de graves
\item[Fuerza el\'ectrica.] Barras sobre soportes giratorio que son frotadas
\item[No fundamental. Fuerza magn\'etica.] Interacci\'on entre imanes
\item[No fundamental. Fuerzas de contacto.] Desplazar un objeto con el dedo. 
\item[No fundamental. Fuerzas de rozamiento.] Interacci\'on el\'ectrica.
\end{description}

\begin{enumerate}
\item {\it Paradoja Mec\'anica.} El cono parece 'caer hacia arriba', pero su centro de gravedad desciende. Por eso es m\'as extra\~no lo que sucede con el tip-top y el huevo cocido (Opcional).
\item {\it Ca\'ida de graves.} La combinaci\'on de la Ley de Gravitaci\'on Universal aplicada a un cuerpo en la superficie de la Tierra y de la Segunda ley de Newton implica que los cuerpos adquieren la misma aceleraci\'on, con independencia de su masa, y que dejados caer de la misma altura tardan lo mismo en alcanzar el suelo.
  \begin{itemize}
  \item Dos bolas, una de vidrio (densidad 2,2 g cm$^{-3}$) y una bola de plomo (densidad 11,0 g cm$^{-3}$), se dejan caer de la misma altura. Discutir el tiempo que tardan en llegar al suelo. Se dejan caer a la vez y casi llegan al mismo tiempo.
  \item Se deja caer cada una de las bolas desde la misma la misma altura (menos de 2 m) y se cronometra con la ayuda de un reloj digital.  Se obtiene aproximadamente 1 s.  Si se dejan caer desde la mitad de altura de obtiene casi lo mismo. Tiempo de reacci\'on.
  \item Se utiliza un dispositivo electr\'onico conectado a un ordenador (electroim\'an, conector, ordenador, interruptor). Se deja caer de unos 2,8 m. Estimar con la ayuda de la ecuaci\'on
$$t = \sqrt{2h\over g}\, ,$$
    (demostrar que esta ecuaci\'on es dimensionalmente correcta) el tiempo que va a tardar.
    Tarda unos 0,75 s
  \item Ca\'ida de graves sobre carrito que se mueve en carril sin rozamiento.
  \end{itemize}

\item {\it Ley de Hooke}
  \begin{itemize}
  \item Con muelles verticales. Se mide la constante el\'astica de un muelle con la ayuda de un dinam\'ometro (de 2 N) y de una regla (de 1 m). Se mide la longitud del muelle en ausencia de fuerzas y con la ayuda del dinam\'ometro y la regla se obtienen fuerza y elongaci\'on y de ah\'i la constante $k$ del muelle. ESta constante se utiliza luego en diferentes experimentos.  

Si se coloca un peso $mg$, y se hace oscilar, el per\'iodo de oscilaci\'on viene dado por
$$T = 2 \pi \sqrt{m\over k}\, .$$
(Demostrar que esta ecuaci\'on es dimensionalmente correcta).

\item Dos muelles horizontales en carril sin rozamiento.  El per\'iodo de oscilaci\'on del carrito, masa 500 g, es de
$$T = 2 \pi \sqrt{m\over 2k}\, .$$ 

\item Resonancia.  Si el carrito con los dos muelles se fuerza con una fuerza sinusoidal, si la frecuencia de la fuerza es la del carrito, entonces entra en resonancia. Se puede amortiguar. 
\end{itemize}

\item {\it P\'endulos}

  \begin{itemize}
  \item P\'endulo matem\'atico. Calcular y medir el per\'iodo de un p\'endulo matem\'atico. 
$$T = 2 \pi \sqrt{l\over g}\, .$$
(Demostrar que la ecuaci\'on es dimensionalmente correcta)
\item P\'endulo f\'isico (Conectado a un ordenador). Calcular y medir el per\'iodo de un p\'endulo f\'isico
$$T = 2 \pi \sqrt{m L^2/3 \over m (L/2) g}\, .$$
(Demostrar que ...)
\item El tentetieso como un p\'endulo f\'isico. 
  \begin{itemize}
  \item Estudiar los dos per\'iodos del tentetieso \'aguila.
  \end{itemize}
\item P\'endulo de gravedad variable (Conectado a un ordenador).  Calcular y medir el per\'iodo de un p\'endulo de gravedad variable:
$$T = 2 \pi \sqrt{l\over g cos \alpha}\, .$$
  Cuando $\alpha = 1,31$ rad (75 $^\circ$), $\cos \alpha = 0,25$ y el per\'iodo debe ser el doble de cuando el p\'endulo se encuentra en vertical.
\end{itemize}
\item P\'endulos resonantes. 
\begin{itemize}
\item P\'endulos resonantes d\'ebilmente acoplados
\item P\'endulo de Wilbeforce. Acoplamiento entre los modos de oscilaci\'on y los modos de torsi\'on.
\end{itemize}
\end{enumerate}


\section{Tercera Ley de Newton}

\begin{enumerate}
\item {\it Coche con globos.} Explicar el movimiento de este coche sobre la base de las presiones no compensadas en el interior del globo. 
\item {\it Helic\'optero el\'ectrico.} Vuela o no vuela dependiendo del sentido de giro. Al cambiar la polaridad del motor, volar\'a cuando antes no volaba. 
\item {\it Helic\'optero de globo.} 
  \begin{itemize}
  \item Explicar el vuelo del mismo y la importancia de los \'angulos 120, 90 y 45 $^\circ$.
  \item Velocidad de ascensi\'on con el globo menos y m\'as hinchado.
  \item Globos menos y m\'as hinchados conectados. Presi\'on del caucho
  \end{itemize}
\item {\it Cilindro con l\'apiz y banda el\'astica.} Si el cilindro est\'a fijo, gira el l\'apiz. Si el l\'apiz est\'a fijo, gira el cilindro. 
\item {\it Coche el\'ectrico con ventilador y vela.} Cuando gira el ventilador, el coche se mueve pues las alas del ventilador golpean el aire y, por reacci\'on, el aire ejerce una fuerza sobre el coche. Si se coloca una vela en otro coche y se aplica el ventilador, el otro coche se mueve. Pero si la vela se coloca enfrente del ventilador del mismo coche, el coche no se mueve, pues el par de fuerzas acci\'on-reacci\'on se aplican sobre el mismo cuerpo.
\item Ca\~n\'on o fusil de Gauss.
\end{enumerate}

\section{Choques el\'asticos}

En el carril de coches sin rozamiento se pueden estudiar choques entre coches de diferentes masas.
\begin{enumerate}
\item
\end{enumerate}


\section{Desaparici\'on y aparici\'on de la energ\'ia mec\'anica}

Adem\'as de las leyes de la Mec\'anica se necesitan leyes adicionales para explicar lo que sucede.
\begin{enumerate}
\item Pelotas que rebotan bastante bien y pelotas que rebotan mal.
\item Debido al rozamiento, toda la energ\'ia mec\'anica desaparece.
\item En el molinete t\'ermico, la energ\'ia mec\'anica aparece gracias al calor. Similitudes y diferencias entre el molinete y el helic\'optero [La unidad, la diversidad y la universalidad de la F\'isica]
\item En la m\'aquina de Her\'on, aparece energ\'ia mec\'anica de rotaci\'on. Relaci\'on con el helic\'optero y la Tercera Ley de Newton. 
\item En el balanc\'in t\'ermico aparece energ\'ia mec\'anica de oscilaci\'on (Opcional).
\item {\it Energ\'ia metaestable.}
  \begin{itemize}
  \item Resorte de caucho (varios)
  \item Resorte arrollador. Teor\'ia de cat\'astrofes.
  \item V\'ortice en una botella.
  \end{itemize}
\end{enumerate}
 
\end{document}
