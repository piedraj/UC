Aula Espacio Tocar la Ciencia.
Curso 2010-2011

Facultad de Ciencias. Nueva Ala. Aula 6
(Junto al Parque de la Ciencia)
Avenida de Los Castros s/n
39005 Santander

Decanato de la Facultad de Ciencias
Aula de la Ciencia
Universidad de Cantabria

De acuerdo con los t�rminos de la colaboraci�n establecida entre el Decanato de la Facultad de Ciencias y el Aula de la Ciencia, se informa de que el Aula Espacio Tocar la Ciencia,  cedida en el marco de esta colaboraci�n, sita en las nuevas dependencias de la Facultad de Ciencias, va a continuar desarrollando sus desarrollar actividades did�cticas de f�sica experimental durante el curso 2010-2011.   

En este espacio se han reunido materiales experimentales relativos a (i) Oscilaciones y Ondas, (ii) �ptica y F�sica Moderna, (iii) M�quinas T�rmicas y (iv) Fluidos.  Estos materiales did�cticos han sido aportados en su mayor parte por el Aula de la Ciencia, dependiente del Vicerrectorado de Difusi�n del Conocimiento y Participaci�n Social, y en menor medida, por el Departamento de F�sica Aplicada.

El aula  se encuentra especialmente preparada para ser utilizada con alumnos de ense�anza secundaria y bachillerato. Los  profesores de Ense�anza Secundaria tendr�n la oportunidad de  reservar  una o varias sesiones para sus propios alumnos, para lo que se ha abierto al efecto la  p�gina en Internet: http://www.loreto.unican.es/Carpeta2011/EspacioTocarLC2010-2011.html.  Durante el curso 2010-2011 se ofertar� a los profesores de Ense�anza Secundaria de Cantabria la posibilidad de reservar el Aula los jueves de 11:30 a 13:30. Estas sesiones estar�n atendidas durante este curso por el Departamento de F�sica Aplicada, por  Julio G��mez, Profesor Titular del Departamento.



Formalizaci�n de Reservas para el Aula Espacio Tocar la Ciencia

Para reservar alguna de las sesiones ofertadas en el   Aula Espacio Tocar la Ciencia durante el curso 2010-2011 se debe 
visitar la p�gina  http://www.loreto.unican.es/Carpeta2011/EspacioTocarLC2010-2011.html
comprobar los d�as que todav�a queden libres, elegir las sesiones que interesen y 
enviar un correo electr�nico a:


guemezj@unican.es

indicando:

(0) que la reserva es para el Aula Espacio Tocar la Ciencia
(i)  la fecha deseada (en principio, dos d�as por colegio, ampliables si van quedando d�as sin reserva),
(ii) nombre del tutor o profesor de los alumnos,
(iii) instituto o colegio,
(iv) n�mero de alumnos (hay un m�ximo de 30 alumnos por sesi�n; si la reserva es por un n�mero menor, otros grupos pueden acomodarse hasta completar los 30)
(v) curso de la ESO o Bachillerato que cursen, y
(vi) un tel�fono de contacto (imprescindible).

Una vez la reserva haya sido formalizada e incorporada a la p�gina de reservas del Aula Espacio Tocar la Ciencia, el profesor solicitante recibir�, en uno o dos d�as,  un correo confirm�ndole que su reserva ha quedado admitida y registrada, lo que �l mismo deber� comprobar en la p�gina de reservas.

Para informaci�n Tfno.  942 201441 (Julio G��mez) y en la Secretar�a del Aula de la Ciencia Tfno. 942 202000 (Cristina Mora)